\documentclass{beamer}
\usepackage[utf8]{inputenc}
\usepackage[T1]{fontenc}
\usetheme{Hannover}

\title{Gitflow: A git workflow}
\usepackage[english]{babel}

\usepackage{graphicx}

\begin{document}
\maketitle{}
\begin{frame}
	
\end{frame}
\begin{frame}
\frametitle{Git}
\begin{itemize}
	\item Git je program za verzioniranje i kontrolu projekata
	\item Razvoj projekata i praćenje verzija moguće je uz samo par linija naredbi
	\item Svaka promjena u projektu pamti se zasebno, te se može inkorporirati ili odbaciti po korisnikovoj želji
	\item Grananje omogućava zaseban razvoj svih takvih promjena bez negativnog utjecaja na sam projekt
\end{itemize}
\end{frame}

\begin{frame}
\frametitle{Workflow}
\begin{itemize}
	\item Workflow je strukturiran set instrukcija koji služi kao vodič pri razvoju projekta
	\item Git također ima nekoliko oblika Workflowa kao što su : 
		\begin{itemize}
			\item Centralized Workflow - Oblik workflowa najsličniji SVN (Apache Subversion Control)
			\item Git Feature Branch Workflow - Razvoj projekta se odvija na zasebnim granama
			\item Git Workflow (Gitflow) - Sličan Git Feature workflow-u ali svaka grana ima svoju ulogu
		\end{itemize}
\end{itemize}	
\end{frame}

\begin{frame}
\frametitle{Gitflow}
\begin{itemize}
	\item Oblik Git Workflow-a koji definira razvoj projekata grananjem (branching)
	\item Gitflow je dizajn Nizozemskog software inžinjera Vincenta Driessena
	\item 
\end{itemize}
\end{frame}
\end{document}